% Author: Dominik Harmim <harmim6@gmail.com>

\documentclass[a4paper, 10pt, twocolumn]{article}

\usepackage[british]{babel}
\usepackage[utf8]{inputenc}
\usepackage[T1]{fontenc}
\usepackage[left=2cm, top=2cm, text={17cm, 25cm}]{geometry}
\usepackage[unicode, colorlinks, hypertexnames=false, citecolor=red]{hyperref}
\usepackage{times}
\usepackage{graphicx}
\usepackage{dirtytalk}
\usepackage[ruled, linesnumbered]{algorithm2e}
\usepackage{lipsum}

\setlength{\algomargin}{2em}

\setlength{\parindent}{0pt}
\setlength{\parskip}{.5 \bigskipamount}


\begin{document}
    \twocolumn[
        \begin{@twocolumnfalse}
            \begin{center}
                {\Large
                    Brno University of Technology \\
                    Faculty of Information Technology \\
                }
                {\includegraphics[width=.4 \linewidth]{img/FIT_logo.pdf}} \\

                {\LARGE
                    Simulation Tools and Techniques \\
                    Project\,--\,Traveling Umpire Problem \\[.4cm]
                }

                {\large
                    Dominik Harmim (xharmi00) \\
                    \texttt{xharmi00@stud.fit.vutbr.cz} \\
                    \today
                }
            \end{center}
        \end{@twocolumnfalse}
    ]


    \section{Introduction}

    In this project, a~selected \emph{operational research} problem is
    described according to the certain published research paper. This
    paper, as well as other resources, are stated in
    Chapter~\ref{sec:resources}. A~given problem is defined in
    Chapter~\ref{sec:tup}. Approaches of a~solution to the problem from the
    paper are re-implemented and experimental results are finally compared
    with the results in the paper.

    The rest of the project documentation is organised as follows. In
    Chapter~\ref{sec:method} there is presented an approach of the solution
    of the problem, and there are also described appropriate methods.
    Re-implementation of the methods is then depicted in
    Chapter~\ref{sec:implementation}. Experimental evaluation is discussed
    in Chapter~\ref{sec:experiments}. Finally, Chapter~\ref{sec:conclusion}
    concludes the documentation and project results.

    \subsection{Resources}
    \label{sec:resources}

    The project is based on the paper~\cite{bendersCutsPaper}. Almost the
    same paper with slight differences was previously published
    in~\cite{bendersCutsCollection}. Further description of the problem
    together with other related publications, detailed results, data sets,
    implemented solvers and validators, and other resources can be found
    in~\cite{tupOnline}.

    \subsection{Traveling Umpire Problem}
    \label{sec:tup}

    By~\cite{bendersCutsPaper, bendersCutsCollection},
    the \emph{Traveling Umpire Problem} is defined as follows. It is
    a~\emph{sports scheduling problem} inspired by the real-life needs of
    the officials of a~sports league. It is all about scheduling the
    \emph{umpires} for tournaments. Each game in a~tournament schedule
    requires an umpire (more precisely the whole umpire crew but this
    fact can be neglected in this problem). The problem of assigning
    the umpires to games in the tournament is subject of several
    \emph{restrictive constraints}. Given these constraints, the primary
    objective is to \emph{minimise the travel} of the umpires among single
    venues.

    Formally, it is given a~\emph{double round-robin tournament} where every
    team plays against all other teams twice. For~$ n $~umpires, there
    are~$ 2n $~teams and $ 4n - 2 $ game slots (rounds), each
    with~$ n $~games. So, it is needed to assign one of the~$ n $~umpires
    to each of these games. And the total travel of all the umpires should
    be as low as possible. Furthermore, the following constraints are
    needed to be satisfied:
    \begin{enumerate}
        \item
            Every game gets an umpire.

        \item
            Every umpire works exactly one game per slot.

        \item
            Every umpire sees every team at least once at the home
            venue of that team.

        \item
            No umpire is in the same venue more than once in any
            $ n - d_1 $ consecutive slots.

        \item
            No umpire sees a~team more than once in any $ \frac{n}{2}
            - d_2 $ consecutive slots.
    \end{enumerate}
    Here, $ d_1 $~and $ d_2 $~are parameters that represent the level of
    constraint required. Setting these to 0 leads to the most constrained
    system. While setting these to~$ n $~and $ \frac{n}{2} $, respectively
    \say{turns off} the corresponding constraint.

    Finding a~\emph{feasible} solution to this problem may be
    challenging. Thus, it makes it difficult to find a~good solution in
    a~reasonable amount of time. The problem is \emph{NP-Hard} and
    exact solution approaches are inefficient in solving large instances.
    In~\cite{bendersCutsPaper, bendersCutsCollection} there are presented
    some exact solution approaches. In particular, there is formulated
    the problem as a~\emph{mixed-integer linear program} and as
    a~\emph{constraint program}. For both cases, it is possible to find
    the \emph{optimum solutions} for small instances. But for larger
    instances, it takes more than dozens of hours to find feasible
    solutions. In the following chapters, there are shown approaches and
    heuristics that help to find good solutions quickly.


    \section{%
        Greedy Matching Heuristic with Benders' Cuts Guided Neighbourhood
        Search
    }
    \label{sec:method}

    In \cite{bendersCutsPaper, bendersCutsCollection} there is considered
    a~hybrid use of \emph{Bender's cuts}, \emph{large neighbourhood
    search}, and a~\emph{greedy matching based heuristic} to find good
    solutions to the Traveling Umpire Problem. By combining these seemingly
    unrelated approaches, it can be found good solutions quickly. In the
    following chapters, there is explained how to generate the Benders'
    cuts during the execution of a~simple greedy heuristic. These cuts
    enforce feasibility requirements that allow the greedy heuristic to
    avoid infeasibilities. These Benders' cuts are embedded in
    a~\emph{very large neighbourhood search} method and by searching this
    neighbourhood, guided by the Benders' cuts, it is obtained a~good
    initial feasible solution which is then further improved by \emph{local
    search}. Combining this approach with some other heuristic described
    later leads to very good solutions.

    A~high level algorithm of a~greedy matching heuristic with the
    Benders' cuts guided neighbourhood search is presented in
    Algorithm~\ref{alg:gbns}. At each time slot, a~greedy matching heuristic
    is used to construct a~schedule. When an infeasibility is recognised
    first a~single step backtracking is tried to resolve the infeasibility.
    If unsuccessful, the Benders' cuts are generated to guide a~large
    neighbourhood search to ensure feasibility and to improve the solution.
    Single steps of the algorithm are explained in the chapters below.

    \begin{algorithm}[hbt]
        Arbitrarily assign umpires to games in 1. slot\;

        \For{$ s = 1 $ \KwTo $ 4n - 2 $}{
            Construct a~bipartite graph $ G = (V, E) $ where the partitions
            are the umpires and the games in~$ s $\;

            Solve a~perfect matching problem on~$ G $ with respect to
            constraints\;

            \If{there is no feasible perfect matching}{
                Backtrack to slot $ s - 1 $ and pick the second
                best matching\;

                Construct a~bipartite graph $ G = (V, E) $ where the
                partitions are the umpires and the games in~$ s $\;

                Solve a~perfect matching problem on~$ G $ with respect to
                constraints\;
            }

            \If{there is no feasible perfect matching}{
                Find all the Benders' cuts\;

                $ Objective \leftarrow (number\ of\ violated\ cuts) *
                cost + (number\ of\ violated\ constraints) * cost +
                (total\ umpire\ travel) $\;

                \While{at least one of the cuts is violated || at least
                one of the constraints is violated}{
                    Do neighbourhood search using 2-ump neighbourhood\;

                    Calculates a~new objective and update a~current
                    solution if the objective is better than the previous
                    one\;
                }
            }
        }

        Do neighbourhood search using 2-ump neighbourhood until all
        constraints are satisfied and the time limit is reached\;

        \caption{%
            Greedy matching heuristic with Benders' cuts guided
            neighbourhood search
        }
        \label{alg:gbns}
    \end{algorithm}

    \subsection{Greedy Matching Heuristic}

    A~simple greedy matching heuristic seamlessly handles constraints~1
    and~2 but it runs into difficulties enforcing constraints~4 and~5.
    Because of that, this approach is then extended to use the Benders'
    cuts and a~neighbourhood search.

    This heuristic builds the umpire schedules starting from the first
    slot and ending at the slot $ 4n - 2 $. For every slot~$ s $, the
    heuristic assigns umpires to games such that all constraints,
    except constraint~3, are satisfied and the best possible assignment
    is made to minimise the total umpire travel at slot~$ s $. The heuristic
    solves a~perfect matching problem on a~bipartite graph in every
    slot~$ s $. The partitions in this graph are the umpires and the games
    in the slot~$ s $. An edge is placed between an umpire~$ u $~and
    a~game $ (i, j) $ (a~game is represented as a~pair $ (i, j) $
    where~$ i $~is the home team and~$ j $~is the away team) if
    constraints~4 and~5 are not violated by assigning~$ u $~to game
    $ (i, j) $ in slot~$ s $, given the assignments from slot~1 to
    $ s - 1 $. The cost of edge $ (u, (i, j)) = Distance(k, i) $. In this
    cost function, $ k $~is the venue that~$ u $~is assigned in slot
    $ s - 1 $ and $ Distance(k, i) $ is the distance between venues~$ k $
    and~$ i $.

    This heuristic often gets stuck at some time slot because there may
    be no feasible perfect matching. In this case, a~set of previous that
    are causing this lack of perfect matching is identified. At least one
    of those previous assignments must be change in order to create
    a~perfect matching. This leads to a~set of Benders' cuts used to
    guide a~large neighbourhood search heuristic. Using this search,
    a~solution that satisfies all the Benders' cuts is found and the
    perfect matching problem for slot~$ s $~can be solved again. It there
    is no feasible perfect matching, it is required to repeat this process.
    The Benders' cuts and the large neighbourhood search are described
    further.


    \subsubsection{Arbitrarily Assignment in the First Slot}

    It was found out that that by arbitrarily assigning the games to
    umpires in the first slot, it is broken the symmetry of the problem
    and the solution quality is then improved and the solution times
    are reduced very significantly. For this reason, this arbitrarily
    assignment was also implemented within this project.

    \subsubsection{Incentive to Assign Unvisited Venues}

    The only constraint that the greedy matching heuristic is unable to
    explicitly handle is constraint~3, because it spans the whole schedule
    and its feasibility can not be ensured within such a~constructive
    heuristic. Thus, this constraint is handled as follows. While forming
    the matching problem at a~slot~$ s $, the cost of an assignment
    $ (u, i) $ is discounted by a~defined discount if umpire~$ u $~has
    never visited venue~$ i $~in the previous slots. This reduction
    guides the heuristic towards assigning the umpires to venues that they
    have not visited yet. So, the possibility of having a~solution that
    violates constraint~3 at the end of the execution of the heuristic is
    reduced. If the resulting solution violates constraint~3, then the
    infeasibility is penalised in the objective function during the course
    of the large neighbourhood algorithm executed afterwards.

    \subsubsection{Backtracking}

    As shows Algorithm~\ref{alg:gbns}, when there is no feasible matching
    at a~slot~$ s $ during the course of the greedy matching heuristic,
    before the Benders' cuts are generated and the neighbourhood search is
    executed, it is first attempt to change the last game-umpire assignment
    to obtain a~feasible matching in slot~$ s $. For that, it is backtracked
    to the previous slot $ s - 1 $ and it is picked the second best
    matching, instead of the best one, and the perfect matching is tried
    for slot~$ s $ again. This backtracking is made at most once at each
    slot.

    \subsection{Very Large Neighbourhood Search with Benders' Cuts}

    An algorithm that starts at some initial solution and iteratively
    moves to solutions in the neighborhood of the current
    solution is called a~neighborhood search algorithm or a~local
    search algorithm. This algorithm is guided by Bender's cuts described
    below.

    \subsubsection{Benders' Cuts}

    Benders' cuts are generated when there is no feasible matching at
    slot~$ s $ during the course of the greedy matching heuristic.
    The Benders' Cut states that at least one of the edges between the
    the umpires and the games that can not be matched because of constraints
    has to be in the graph. Further information about these cuts can be
    found in~\cite{bendersCutsPaper}.


    \section{Implementation}
    \label{sec:implementation}

    The described algorithm has been implemented in Python programming
    language. At least, it is required Python version 3.6. Python libraries
    NumPy\footnote{Python library for mathematics computation
    NumPy\,--\,\url{https://numpy.org}.} and SciPy\footnote{Python library
    for mathematics computation SciPy\,--\,\url{https://www.scipy.org}.}
    was used for mathematics computation. Required versions of these
    libraries are specified in file \texttt{requirements.txt}. The
    implementation consists of several modules described in the paragraphs
    below.

    The module \texttt{main} severs an entry point of the application.
    It reads an input file with the specification of the problem and
    it executes the greedy matching heuristic.

    Modules \texttt{inp} and \texttt{out} contains functions for reading
    and parsing input files with the specification of the problem, and
    functions for writing solutions to output files in the appropriate
    format.

    A~class \texttt{Tup} in module \texttt{tup} represents the Traveling
    Umpire Problem. The class encapsulates the specification of the problem
    as well as its input parameters. Moreover, it contains operations such
    as computation of the total umpire travel or computation of violations
    of single constraints. All of the methods are properly commented in
    the source code.

    The most interesting is module \texttt{gmh} which contains five main
    functions. Function \texttt{gmh} the greedy matching heuristic
    algorithm. It tries to find a~perfect matching in every slot. In the
    case that there is no feasible perfect matching, the Benders' cuts
    are used to guide the large neighbourhood search heuristic. Function
    \texttt{benders\_cuts} computes all the Benders' cuts in a~given slot.
    Function \texttt{benders\_violations} calculates the number of
    violated Benders' cuts in a~current solution. The very large
    neighbourhood search algorithm uses the Benders' cuts is implemented
    in function \texttt{neigh\_search}. Finally, function
    \texttt{neigh\_search\_objective} computes the value of the objective
    function which is used during the large neighbourhood search.

    \subsection{Created Tool Usage}

    The created tool is capable of read an instance of the problem and
    a~solution. Format of an input instance is described in
    Section~\ref{sec:datasets}. Experimental problem instances are in
    directory \texttt{input/}. Output of the tool is a~single line
    indicating the umpire assigned for each game as follows:
    $$
        A_{11}, A_{12}, \ldots, A_{1g}, A_{21}, A_{22}, \ldots, A_{2g},
        \ldots, \ldots,
    $$
    $$
        A_{r1}, A_{r2}, \ldots, A_{rg}
    $$
    where $ A_{rg} \in \{1, 2, \ldots, U\} $ denotes the umpire assigned
    to game $ g \in \{1, 2, \ldots, U\} $ of round $ r \in \{1, 2,
    \ldots, R\} $. $ U $~is the number of umpires and~$ R $~is the number
    of rounds. The solution will be printed in the appropriate file to
    directory \texttt{output/}.

    There is a~prepared \texttt{Makefile} for running the program. It can
    be run using the following command: \texttt{make run INST=inst
    D1=d1 D2=d2 LIMIT=limit}, where \texttt{inst} is a~name of an instance
    of the problem, \texttt{d1} is the parameter~$ d_1 $~of 4. constraint,
    \texttt{d2} is the parameter~$ d_2 $~of 5. constraint, and \texttt{limit}
    is the time limit of the execution. E.g., \texttt{make run INST=umps4
    D1=0 D2=0 LIMIT=30} executes the program with an instance with~4
    teams stored in input directory with $ d_1 = d_2 = 0 $ and with the
    time limit 30 minutes.


    \section{Experimental Evaluation}
    \label{sec:experiments}

    This chapter describes used data sets, discusses executed experiments,
    and compares the results with the paper.

    \subsection{Data Sets Description}
    \label{sec:datasets}

    Data sets are available in~\cite{tupOnline}. It also can be found
    in directory \texttt{input/}. It was used the same problem instances
    as the instances used in the paper, so it can be easily compared.
    A~format of the instances is described in file
    \texttt{input/input-format.pdf}.

    \subsection{Experiments Results}

    Table~\ref{tab:exp1} show results of execution of the program on
    smaller problem instances from 4~teams to 10~teams. The instances
    followed by a~letter A, B, or C are the same as the original with
    the distances matrix is changed. In the Table~\ref{tab:exp2} there
    can be seen experiments with larger instances.

    Experiments show that results of the implemented algorithm within this
    project are comparable with the results in the paper. These algorithm
    has been run for several minutes. It is likely that longer run would
    lead to more optimal results.

    \begin{table}[hbt]
        \centering

        \begin{tabular}{|l||r|r|r|}
            \hline
            Instance & Optimal Dist. & Paper & \textbf{Harmim} \\
            \hline \hline

            4 & 5,176 & 5,176 & \textbf{5,176} \\ \hline
            6 & 14,077 & 14,077 & \textbf{14,077} \\ \hline
            6A & 15,457 & 15,457 & \textbf{15,735} \\ \hline
            6B & 16,716 & 16,716 & \textbf{16,716} \\ \hline
            6C & 14,396 & 14,396 & \textbf{14,396} \\ \hline
            8 & 34,311 & 34,311 & \textbf{34,311} \\ \hline
            8A & 31,490 & 31,490 & \textbf{31,490} \\ \hline
            8B & 32,731 & 32,731 & \textbf{32,731} \\ \hline
            8C & 29,879 & 29,879 & \textbf{29,879} \\ \hline
            10 & 48,942 & 49,528 & \textbf{53,898} \\ \hline
            10A & 46,551 & 46,822 & \textbf{49,259} \\ \hline
            10B & 45,609 & 45,609 & \textbf{51,920} \\ \hline
            10C & 43,149 & 43,533 & \textbf{45,396} \\ \hline
        \end{tabular}

        \caption{%
            Comparison of experimental results of smaller instances
            executed in the paper and in this project (Harmim)
        }
        \label{tab:exp1}
    \end{table}

    \begin{table}[hbt]
        \centering

        \begin{tabular}{|l||c|c|r|r|}
            \hline
            Instance & $ n - d_1 $ & $ \frac{n}{2} - d_2 $ & Paper
            & \textbf{Harmim} \\ \hline \hline

            14 & 6 & 3 & 176,290 & \textbf{178,697} \\ \hline
            14 & 5 & 3 & 167,146 & \textbf{181,675} \\ \hline
            14A & 6 & 3 & 172,764 & \textbf{176,445} \\ \hline
            14A & 5 & 3 & 163,978 & \textbf{170,624} \\ \hline
            14B & 6 & 3 & 179,278 & \textbf{175,161} \\ \hline
            14B & 5 & 3 & 160,717 & \textbf{172,492} \\ \hline
            14C & 6 & 3 & 172,243 & \textbf{183,794} \\ \hline
            14C & 5 & 3 & 168,970 & \textbf{173,963} \\ \hline
            16 & 7 & 2 & 166,274 & \textbf{181,548} \\ \hline
            16A & 7 & 2 & 177,857 & \textbf{189,470} \\ \hline
            16B & 7 & 2 & 184,923 & \textbf{197,952} \\ \hline
            16C & 7 & 2 & 181,013 & \textbf{202,693} \\ \hline
        \end{tabular}

        \caption{%
            Comparison of experimental results of larger instances
            with less restricted constraints executed in the paper and
            in this project (Harmim)
        }
        \label{tab:exp2}
    \end{table}


    \section{Conclusion}
    \label{sec:conclusion}

    Within this project, algorithm from~\cite{bendersCutsPaper}
    for solving the Traveling Umpire Problem has been successfully
    implemented and experimental evaluated.


    \bibliographystyle{englishiso}
    \renewcommand{\refname}{Bibliography}
    \bibliography{doc}
\end{document}
